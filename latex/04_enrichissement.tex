\chapter{Enrichissement des clusters}

Une fois la méthode de clustering choisie (MeanShift), il reste à catégoriser les clusters obtenus afin d'en extraire des informations utiles.

Plusieurs méthodes sont possibles à partir du jeu de donnée que nous avons~:

\begin{enumerate}
    \item \textbf{Établir une description des clusters à partir du nombre d’occurrences des hashtags qui y sont présents~:}

    Permet d'avoir une première idée sur ce que peut représenter un cluster. Mais présente plusieurs défis, notamment l'élimination de hashtags
    qui sont présents dans beaucoup d'autres clusters, afin de ne garder que les hashtags significatifs.

    \item \textbf{Forcer un 3IF à annoter manuellement tous les clusters~:}

    Plus pertinent que l'utilisation des hashtags (et consommant moins de cycles CPU), mais apparemment interdit par le règlement intérieur de l'INSA.

    \item \textbf{Rechercher les lieux connus proches des clusters~:}

    C'est la solution que nous avons retenu, en utilisant pour cela l'API \textit{Google Places}.
    Cette API permet de retrouver des \textit{places} (monuments, restaurants, hôtels, quartiers, ...) à partir des coordonnées d'un point
    et d'un rayon maximal de recherche.

    Nous avons donc utilisé cette API pour trouver les endroits les plus proches des centres de nos cluster, et nous avons choisi
    l'endroit le plus connu parmi ces derniers, ce qui donne de très bons résultats (sur la majorité des clusters que nous avons testés)

    Nous aurions aussi pu utiliser les hashtags (en les comparant à la description des lieux donnés par l'API Google Places) afin de
    choisir l'endroit le plus pertinent.
\end{enumerate}
