
% --------------------------------------------------------------- CONFIGURATIONS

\documentclass[a4paper,12pt,final,oneside]{book}

\usepackage{rapport}
\usepackage{float}


% -------------------------------------------------------------- META: CONSTANTS

\newcommand{\reporttitle}{Data Mining}
\newcommand{\enseignants}{Jean-Francois~\textsc{Boulicaut}\\ Mehdi~\textsc{Kaytoue}}
\newcommand{\reportauthor}{Guillaume~\textsc{Abadie}\\ Ahmed~\textsc{Kachkach}}
\newcommand{\reportsubject}{Livrable de projet}
\newcommand{\stagetopic}{D\'ecouverte de points d'int\'er\^ets}
\newcommand{\dateperiod}{du 24 fevrier au 10 mars 2014}
\newcommand{\HRule}{\rule{\linewidth}{0.5mm}}
\setlength{\parskip}{1ex} % Espace entre les paragraphes

\hypersetup{
	pdftitle={\reporttitle},%
		pdfauthor={\reportauthor},%
		pdfsubject={\reportsubject},%
		pdfkeywords={INSA Lyon} {Data Mining}
}

\title{\reporttitle}
\author{\reportauthor}
%\setcounter{tocdepth}{4}


% ------------------------------------------------------------------------- FILE

\begin{document}


    % ------------------------------------------------------------------- HEADER

	\renewcommand{\chaptername}{} %\renewcommand{\thechapter}{}
	\renewcommand{\contentsname}{Sommaire}

	\pagestyle{empty}
	\pagenumbering{Roman}


    % ------------------------------------------------------------ HEADER: TITLE

	% Inspiré de http://en.wikibooks.org/wiki/LaTeX/Title_Creation
\begin{center}
	\begin{minipage}[t]{0.48\textwidth}
	  \begin{flushleft}
	    \includegraphics [width=40mm]{images/logo_INSA.png} \\[0.5cm]
			INSA Lyon\\
			20, avenue Albert Einstein\\
			69621 Villeurbanne Cedex
	  \end{flushleft}
	\end{minipage}
	\begin{minipage}[t]{0.48\textwidth}
	  \begin{flushright}
	    %\includegraphics [width=60mm]{images/logo_Passau.jpg} \\[0.5cm]
	    %Universität Passau\\
		%Innstraße, 3\\
		%	D-94032 Passau
	  \end{flushright}
	\end{minipage} \\[2cm]

	\textsc{\Large \reportsubject}\\[0.3cm]
	\HRule \\[0.4cm]
	{\Huge \bfseries \reporttitle}\\[0.3cm]
	{\LARGE \bfseries «~\stagetopic~»}\\[0.3cm]
	{\Large \dateperiod}\\[0.4cm]
	\HRule \\[1cm]

	\includegraphics [scale=0.35]{images/voitureLagaffe.jpg} \\[0.7cm]
	\begin{minipage}[t]{0.4\textwidth}
	  \begin{flushleft} \large
	    \emph{Binôme~:}\\
	    \small \reportauthor
	  \end{flushleft}
	\end{minipage}
	\begin{minipage}[t]{0.5\textwidth}
	  \begin{flushright} \large
	    \emph{Enseignants~:} \\
	    \enseignants
	  \end{flushright}
	\end{minipage}

	\vfill
	\footnotesize{Année scolaire 2013-2014}
\end{center}



    % --------------------------------------------------- HEADER: CONFIGURATIONS

	\sloppy          % Justification moins stricte : des mots ne dépasseront pas des paragraphes

    \frontmatter
		\pagestyle{empty}
		\tableofcontents
		\addtocontents{toc}{\protect\thispagestyle{empty}}

	\mainmatter
	\pagestyle{headings}

	\renewcommand{\chaptermark}[1]{\markboth{\MakeUppercase{\chaptername\ \thechapter.\ #1}}{}}
	\renewcommand{\sectionmark}[1]{\markright{\thesection{} #1}}


    % ------------------------------------------------------------------ CONTENT

    \chapter{Application}

    Nous avons exploité une base de données de \~80.000 photos géolocalisées extraites de Flickr dans le cadre d'une découverte
    non-assistée de points d'intêrets, avec catégorisation et filtrage de ces derniers. (comme c'est le cas sur Google Maps, par exemple)

    Pour cela, nous nous sommes d'abord familiarisés avec les données, décidé des données à garder, nettoyé le document (csv) contenant ces dernières et expérimenté quelques méthodes de clustering avec Knime.

    Nous sommes ensuite passé en production avec un stack Python~: pandas pour le traitement et la sélection des données, numpy et scikits-learn pour le clustering et flask pour servir notre application web.

    Nous discuterons toutes ces étapes avec plus de détails dans les sections à suivre, mais nous vous invitons vivement à d'abord
    tester \href{http://188.226.171.22/}{\textbf{notre application web}}. Vous pouvez aussi voir \href{http://www.youtube.com/watch?v=SGGsvbjWYY0}{\textbf{ce court screencast}} faisant la démonstration de ses principales fonctionnalités.

    Les points d'intêrets sont représentés par des cercles dont la taille est proportionelle à l'intêret qu'ils représentent (= nombre de photos liées).

    Voici aussi quelques screenshots de l'application~:

    \begin{figure}[H]
        \centering
        \includegraphics[scale=0.25]{../screenshots/ui-global.png}
        \caption{Interface globale de l'application}
        \label{diagram:ui-global}
    \end{figure}


    \begin{figure}[H]
        \centering
        \includegraphics[scale=0.25]{../screenshots/ui-info-cluster.png}
        \caption{Informations sur un point d'intêret, ici la Basilique de Fourvière}
        \label{diagram:ui-info-cluster}
    \end{figure}

    \begin{figure}[H]
        \centering
        \includegraphics[scale=0.25]{../screenshots/ui-filter-museum.png}
        \caption{On peut filtrer les points d'intêret par type. Par exemple chercher un musée...}
        \label{diagram:ui-filter-museum}
    \end{figure}

    \begin{figure}[H]
        \centering
        \includegraphics[scale=0.25]{../screenshots/ui-filter-park.png}
        \caption{... ou chercher un parc.}
        \label{diagram:ui-filter-park}
    \end{figure}

    \pagebreak


	\chapter{Chargement et nettoyage des données}

\section{Prototypage et découverte des données}
    % TODO : Garder cette partie ici, ou faire un chapitre préliminaire où on %
    % parle du fait d'utiliser Knime puis de passer sur python ?              %
    Nous avons dans un premier temps utilisé Knime pour exploiter la base de données,
    et tester les algorithmes des clustering.

    Pour cela, nous avons d'abord configuré le nœud de lecture de fichiers CSV~: choix du
    séparateur de colonnes (tabulation, dans notre cas), autorisation des \textit{shortlines},
    gestion des valeurs manquantes, ...

\section{Types des données}
    Nous avions des types parasites dans les données. En effet nous
    avions des colonnes considérées comme ``String'' alors qu'elles étaient
    censées être des ``Double''. Ceci est dû à des valeurs parasites (la fameuse
    valeur de latitude ``trolilol'', notamment).

\section{Unicité des valeurs}
    Nous avions aussi détecté des problèmes d'unicité des valeurs. Par
    exemple, il apparaît que l'ID des photos n'est pas unique.
    Ceci peut être dû au fait que la récupération (\textit{scrapping}) de telles données
    se fait souvent de manière parallélisée, et qu'une même photo peut ainsi être récupérées plusieurs fois.

    En partant de l'hypothèse que ces duplicatas représentent les mêmes
    informations, on peut arbitrairement garder un élément par id de photo.

    Une deuxième solution, plus rigoureuse, est de créer un nouvel id garantissant
    l'unicité des informations (c'est à dire l'ensemble des colonnes pertinentes: id de
    la photo, id de l'utilisateur, date, tags et légende). C'est cette solution que nous avons choisi d'appliquer.

\section{Validité des données}
    L'élimination des duplicatas et la vérification du type des données ne suffit pas à garantir la validité de ces dernières.

    Par exemple, bien que l'année de prise d'une photo soit un entier, il est bien possible que
    celle ci soit incohérente (pour de multiples raisons: corruption de données, troll d'enseignants, etc.) et
    altère la valorisation des données (par exemple en perturbant le clustering).

    Par exemple une photo prise en 1780, avant l'invention de l'appareil photo ou
    bien une photo ayant été prise... dans 10 ans !

    Ces valeurs incohérentes peuvent parfois être conservées, ou être ``corrigées'' en les remplaçant par une moyenne globale (ou locale,
    calculée sur les éléments semblables). Dans notre cas, nous avons choisi d'éliminer les photos présentant des incohérences au vu de
    l'amplitude de celles ci, et du faible nombre de photos concernées.

    Voici les critères de validité retenus~:

    \subsection{Année de la prise de la photo}
        Ainsi, nous avons décidé de filtrer tout les photos ayant été prises
        avant l'an 2000 ou dans le futur.

        \begin{figure}[h]
            \centering
            \includegraphics[scale=0.35]{../screenshots/year_id_before.png}
            \caption{données avant filtrage}
            \label{diagram:year_id_before}
        \end{figure}

        \begin{figure}[h]
            \centering
            \includegraphics[scale=0.35]{../screenshots/year_id_after.png}
            \caption{données apr\`es filtrage}
            \label{diagram:year_id_after}
        \end{figure}

    \pagebreak
    \subsection{Mois et jour}

        Nous n'avons pas poussé la validation jusqu'à vérifier que la chaque jour ne dépassait pas la longueur de chaque mois, mais nous avons décidé d'éliminer toutes les photos prises à un jour > 31 (ou < 1) et les mois > 12 (ou < 1).

        \begin{figure}[h]
            \centering
            \includegraphics[scale=0.35]{../screenshots/month_day_before.png}
            \caption{données avant filtrage}
            \label{diagram:month_day_before}
        \end{figure}

        \begin{figure}[h]
            \centering
            \includegraphics[scale=0.35]{../screenshots/month_day_after.png}
            \caption{données apr\`es filtrage}
            \label{diagram:month_day_after}
        \end{figure}

    \pagebreak
    \subsection{Jour et heure}

        On aimerait tous que les jours fassent plus de 24 heures, mais pour ce projet là nous avons malgré tout éliminé les valeurs plus grandes que 23.

        \begin{figure}[h]
            \centering
            \includegraphics[scale=0.27]{../screenshots/day_hour_before.png}
            \caption{données avant filtrage}
            \label{diagram:day_hour_before}
        \end{figure}

        \begin{figure}[h]
            \centering
            \includegraphics[scale=0.35]{../screenshots/day_hour_after.png}
            \caption{données apr\`es filtrage}
            \label{diagram:day_hour_after}
        \end{figure}

    \pagebreak
    \subsection{Coordonnées GPS}

        Probablement pour des raisons de conversion (ou de troll pur et simple de la part de notre source de données),
        un point était situé aux coordonnées (0, 0). Nous avons donc préféré éliminer ce point individuellement au lieu
        de restreindre les valeurs de latitude et de longitude (de ce fait, nous avons aussi gardés des points éloignés
        du centre de Lyon, mais pertinents dans notre étude)

        \begin{figure}[h]
            \centering
            \includegraphics[scale=0.22]{../screenshots/geographic_before.png}
            \caption{Coordonnées GPS avant filtrage}
            \label{diagram:geographic_before}
        \end{figure}

        \begin{figure}[h]
            \centering
            \includegraphics[scale=0.22]{../screenshots/geographic_after.png}
            \caption{Coordonnées GPS apr\`es filtrage}
            \label{diagram:geographic_after}
        \end{figure}

	\chapter{Clustering}

Nous avons expérimentés plusieurs méthodes de clustering pour extraire des points d'intêrets à partir des données
des quelques 80 000 photos que nous possédons.

Nous présentons dans cette partie les méthodes que nous avons utilisées, leur pertinence dans le cadre de ce projet, et
autres avantages et/ou inconvénients.

Nous avons principalement fais un clustering basé sur la latitude/longitude. Nous montrons néanmoins (dans la partie K-Means) qu'un clustering sur d'autres critères, par exemple le mois de prise de la photo, ne donne pas de résultats satisfaisant.

\section{Clustering hiérarchique}
    Cette méthode, bien qu'utile quand on ne connaît pas le nombre de clusters
    attendu (comme c'est le cas dans ce projet), ne passe pas à l'échelle~:
    son exécution sur la totalité du jeu de donnée provoque un dépassement de mémoire.

    Nous avons malgré tout réalisé un échantillonnage aléatoire (de 1,000 lignes)
    de notre jeu de données sur lequel on a effectué un \textit{clustering hiérarchique}
    afin d'avoir une idée sur le nombre de clusters qui pouvaient être extraits.

    \begin{figure}[H]
        \centering
        \includegraphics[scale=0.3]{../screenshots/hierarchical_clustering_1000_samples.png}
        \caption{Résultat du clusturing hiérarchique sur 1000 échantillons}
        \label{diagram:hierarchical_clustering_1000_samples}
    \end{figure}

    Mais même sur un jeu de données aussi restreint (par rapport aux données
    originales), le résultat était difficilement exploitable (car très dense),
    et le fait que les données exploitées représentaient moins de $5\%$ des
    données originales faisait que ce clustering était fort instable (structure
    variant sur des samplings différents).

\section{K-Means}
    Bien que plus rapide à l'execution que le clustering hiérarchique (et moins demandant en ressources mémoire),
    le clustering K-Means n'est pas des plus adaptés pour ce projet~: il demande de connaitre le nombre de clusters et
    donne des clusters parfois répartis sur de grandes zones et moins denses que ce que donnent d'autres méthodes (notamment Mean Shift).

    \begin{figure}[H]
        \centering
        \includegraphics[scale=0.25]{../screenshots/kmeans_geographic.png}
        \caption{Résultat du clustering K-Means}
        \label{diagram:kmeans_geographic}
    \end{figure}

    Nous avons aussi essayé de réaliser des clusters en nous basant sur le mois de prise des photos, afin de montrer certaines tendances (quartier plus fréquentés pendant l'été, par exemple) mais les résultats étaient peu satisfaisants (pas de correlation particulière entre le mois de prise de la photo et sa position).

    \begin{figure}[H]
        \centering
        \includegraphics[scale=0.28]{../screenshots/kmeans_month.png}
        \caption{Résultat du clustering K-Means basé sur la date de prise des photos}
        \label{diagram:kmeans_month}
    \end{figure}


\section{DBScan}
    DBScan cumule les tares de K-Means et du clustering hiérarchique : très lent à s’exécuter (avec dépassement mémoire
    si on utilise la totalité du jeu de données) et donnant des clusters très dispersés, même bien moins denses que ceux de K-Means.


    \begin{figure}[H]
        \centering
        \includegraphics[scale=0.3]{../screenshots/dbscan_geographic.png}
        \caption{Résultat du clustering avec DBScan}
        \label{diagram:dbscan_geographic}
    \end{figure}


\section{Mean Shift}

    \textbf{TODO : finir cette partie}
    \chapter{Enrichissement des clusters}

Une fois la méthode de clustering choisie (MeanShift), il reste à catégoriser les clusters obtenus afin d'en extraire des informations utiles.

Plusieurs méthodes sont possibles à partir du jeu de donnée que nous avons~:

\begin{enumerate}
    \item \textbf{Établir une description des clusters à partir du nombre d’occurrences des hashtags qui y sont présents~:}

    Permet d'avoir une première idée sur ce que peut représenter un cluster. Mais présente plusieurs défis, notamment l'élimination de hashtags
    qui sont présents dans beaucoup d'autres clusters, afin de ne garder que les hashtags significatifs.

    \item \textbf{Forcer un 3IF à annoter manuellement tous les clusters~:}

    Plus pertinent que l'utilisation des hashtags (et consommant moins de cycles CPU), mais apparemment interdit par le règlement intérieur de l'INSA.

    \item \textbf{Rechercher les lieux connus proches des clusters~:}

    C'est la solution que nous avons retenu, en utilisant pour cela l'API \textit{Google Places}.
    Cette API permet de retrouver des \textit{places} (monuments, restaurants, hôtels, quartiers, ...) à partir des coordonnées d'un point
    et d'un rayon maximal de recherche.

    Nous avons donc utilisé cette API pour trouver les endroits les plus proches des centres de nos cluster, et nous avons choisi
    l'endroit le plus connu parmi ces derniers, ce qui donne de très bons résultats (sur la majorité des clusters que nous avons testés)

    Nous aurions aussi pu utiliser les hashtags (en les comparant à la description des lieux donnés par l'API Google Places) afin de
    choisir l'endroit le plus pertinent.
\end{enumerate}


    % ------------------------------------------------------------------- FOOTER
\end{document}
